\begin{document}
\section{Conclusion}
In the result we have seen that this model of doing optimisations can find interesting
optimisations that are difficult to find using normal optimisations techniques\cite{santos}.
For example we showed that the following snippet would be optimised to true even
when we don't know what the values of either $x$ nor $y$ are.

\begin{verbatim}
x + y == y + x
\end{verbatim}

This is akin to a proof search, we find terms that are equal by applying axioms
and we get a proof\footnote{We can only call it a proof if the axioms really are 
consistent with our equivalence relation.} that $x + y \sim y + x$. This proof
is used by another proof that says that semantical equal terms will evaluate to 
True by the == operator in the language.

We haven't yet tackled the issue with scope since it contains complications
with variable-bindings that can flee from the scope. So when considering the classes
we must also take into account in which scope the equivalence is in. We chose not
to go further into that direction and instead focused on languages that doesn't
contain any variable bindings.

There are a couple of languages that fall into this category, we have used a simple
arithmetical language but more would be possible (and a simple to implement). For
example simple propositional logic with simple laws to describe equal propositions,
this could be useful in finding smaller representations for formulas. Similar
relational algebra could be modeled and used to optimise static queries to databases.

%* Our language was a bit to simple to show really interesting things?
%* compare our axioms to hard coded rules and why this approch is better/different.
%* We have shown that in this model we can find optimisations that are not easy
%to find otherwise (x + y == y + x) 
%* We have a fast implementation, for languages that doesn't contain scope 
% + relationalalgebras
% + mathematical formulas (without forall/exists)
% + propositional logic
 \end{document}