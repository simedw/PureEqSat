\begin{document}
\section{Implementation}
In this section we will take a closer look at the different aspects that relate
to the implementation of the equality saturation. We will start of with how
we store the classes, and then we will explain how we find the equivalences by 
the use of axioms. Finaly we talk about how we choose the best representation in
a given class.
\subsection{How do we keep track of the classes?}
To keep track of the the classes we must have an efficient data-structure to store
them in. The different operations we may perform on classes is of the followin:

\begin{description}
  \item[makeClass] Given a term, we create a class where it belongs to.
  \item[union]     Given two classes we may find that they are in fact equal, then
                   we need to merge them.
  \item[addElem]   Sometimes we find new terms that we know in which class they
                   should belong to.
\end{description}

Due to these requirements we have opted for a union-find structure and acompaning
algorithms. We have added some structure to it, for example we store all the terms
in the root node. We represent this by the following data:

\begin{verbatim}
data EqRepr s
  = Root ID Rank (EqData s)
  | Node (IORef (EqRepr s))
\end{verbatim}

Here we can plug in the data we need in the \verb|EqData| data type, we will have
the elements of this class and also have other \verb|EqRepr| that depend on that
class. So if $\#x \lhd \#y$ then $\#x$ will store $\#y$ which we use when we update
$\#x$ to signal that it could be beneficial to check the rules against $\#y$.

\subsection{Pattern-Matching on terms to see find axioms to apply}
* * looking throw every term for match in the class. (hint callback future work) 
* dirty-bit work
  * depth of change in <|
\subsection{Cost function}
The final step in our optimisation is to pick out the best term, we do this by
storing the best represetant from each class. To decide which term that we should
consider best we have a cost function that given a term gives us a value how
much it cost to evaluate that function. To simplify further we put the following
restrictions on it.

\begin{itemize}
  \item Forall terms $t$, $f(\blacksquare) \le f(t)$
  \item If $t$ and $t'$ are terms such that $f(t) \le f(t')$ then forall $c$,
         $f(t \otimes c) \le f(t' \otimes c)$
\end{itemize}

The first point says that if we have a value we don't need to compute any compound
expression. The second is requirement is so that we only need to consider the value
of $f$ for the best value of the subterm, i.e. we only need to check against the
best representant of the subterms class.

These requirements almost make it possible to solve the problem with using a 
straight-forward dynamic programming technique. But as we have seen before we can
have $x \lhd x$ and even worse we can have cycles. The first point helps us in most
of the $x \lhd x$ but to solve the more general cycles we have opted for a simple 
solution, this mostly due to time constraints.

When we calculate the result for the differnt classes, we also keep track of all
the classes we currently are calculating for that aren't finished yet. If we encounter
a term that depends on one of this classes we simply disregard it. This of course
is not an optimal solution. What we really would wanna do is put it on the side,
and when the offending classes have gotten theire priliminary values, we check if
any of the terms we put on the side would get a better result. In that case we use that
value and recalculate until we get a fixed point.

We haven't tested the final remark and we don't know if it really will work, but
is seems reasonable to assume that a fixed point would be found since we only have
a finite number of combinations of terms and in each iteration we get better representants.
\end{document}